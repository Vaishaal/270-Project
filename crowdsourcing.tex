    Many machine learning algorithms rely on having accurate ground truth for relatively complicated data sets. These ``ground truth" labels are usually delegated to humans on some sort of system like Amazon's Mechanical Turk. Unfortunately due to the circular nature of the problem there is no way to verify the accuracy of these labels. Much of the previous work in this field revolved around the David-Skene Estimator \cite{dawid1979maximum}. The Dawid-Skene estimator has been widely used for inferring the true labels from the noisy labels provided by non-expert crowdsourcing workers. However, since the estimator maximizes a non-convex log-likelihood function, it is hard to theoretically justify its performance. Moreover the iterative nature of EM gives no bound on the speed of convergence.  Recent work by Zhang et al. \cite{zhang2014spectral} presents a new spectral method for ground truth label generation by using spectral methods to estimate the confusion matrix for the labelers, and provides a good theoretical bound on the rate of convergence. In fact in the papers results section, it claims it reaches very close to a global optimum in one iterate of EM. For the sake of simplicity we will consider only the two class classification case, the generalization to k classes is quite natural.

\subsection{Problem Statement}
Consider an experiment where $M$ labelers assign labels to $N$ samples from the set: $\left\{[0,1],[1,0]\right\}$ (negative,positive). I will refer to these vectors as $N$ and $P$ respectively. Note though our labels represent scalars we represent them by the standard basis vectors of $\mathcal{R}^{2}$. This will simplify our later analysis. We will denote labeler $m$'s label of sample $n$ as $z_{mn}$. The true label $y_{i}$ of item
$i \in \left\{0,1....N\right\} $ is assumed to be sampled from a probability
distribution $Pr(y_{i} = 0)$ = $w_{0}$ where $w_{0}$ is the prior on label 0.  We define $W$ to be a diagonal $2 \times 2$ matrix:
\begin{align}
\begin{bmatrix}
w_{0} & 0 \\
0     & w_{1}
\end{bmatrix}
\end{align}

Naturally $w_{0} + w_{1} = 1$. \\

We also assume that each of the labelers are conditionally independent given the true label. More formally:

$$Pr(z_{ij} = X | y_{j} = X)Pr(z_{hj} = X | y_{j} = X) = Pr(z_{ij}z_{hj} | y_{j} = X)$$

Where $i \neq h$.

Though this assumption may not hold for all instances of the labelling problem, it greatly simplifies our solution. We also assume that the vendors are more likely to get a sample correct than incorrect.

Finally we assume each worker is associated with a $2 \times 2$ confusion matrix. Where $(l,c)$-th entry represents $P(\text{Worker Label} = l| \text{True Label} = c)$. We will denote the confusion matrix for worker $j$ as  $C_{j}$.

Now what we want is to estimate $y_{i}$ for all $i \in \left\{0,1....N\right\}$. And estimate $C_{j}$ for all $j \in \left\{0,1....M\right\}$. We would also like to estimate $W_{0}$ and $W_{1}$



\subsection{Skene Estimator}

\subsection{Expectation Maximization Algorithm}
\subsubsection{Analysis}

\subsection{Spectral Initialization}
