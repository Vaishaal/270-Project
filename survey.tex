\documentclass{article}
\usepackage[utf8]{inputenc}
\usepackage[margin=1in]{geometry}
\usepackage{amsmath,mathtools,amssymb}

\setlength\parindent{0pt}
\title{270 Project}
\author{Achal Dave, Vaishaal Shankar}
\date{}

\DeclarePairedDelimiter\ceil{\lceil}{\rceil}%
\DeclarePairedDelimiter\floor{\lfloor}{\rfloor}%
\DeclarePairedDelimiter\abs{\lvert}{\rvert}%
\DeclarePairedDelimiter\norm{\lVert}{\rVert}%
\newcommand{\E}{\mathbb{E}}
\newcommand{\R}{\mathbb{R}}
\newcommand{\cA}{\mathcal{A}}
\newcommand{\cL}{\mathcal{L}}
\newcommand{\st}{\text{s.t.}}
\newcommand{\Var}{\mathrm{Var}}
\newcommand \textlcsc[1]{\textsc{\MakeLowercase{#1}}}

\DeclareMathOperator*{\argmin}{arg\!\min}
\DeclareMathOperator*{\argmax}{arg\!\max}
% for toggles
\usepackage{etoolbox}
\usepackage[usenames]{color}
\usepackage[usenames, dvipsnames]{xcolor}

\newtoggle{comments}
% \togglefalse{comments}
\toggletrue{comments}
\iftoggle{comments} {
    \newcommand {\achal}[1]{{\color{Orange}\bf{(AD: #1)}\normalfont}}
    \newcommand {\vaishaal}[1]{{\color{Blue}\bf{(VS: #1)}\normalfont}}
}{
    \newcommand {\achal}[1]{}
    \newcommand {\vaishaal}[1]{}
}


\begin{document}


\maketitle
\tableofcontents
\clearpage


\section{Introduction}
This survey focuses on spectral methods applied to machine learning,
specifically their use in problems related to large datasets. Spectral methods
exploit the abundance of prior work in linear algebra to solve solve various
classes of problems. Recent advances in the efficiency of randomized linear
algebra algorithms makes spectral methods increasingly attractive particularly
when dealing with problems with large weakly structured data sets. Specifically
we will be looking at spectral clustering and recent work on spectral methods
for crowd labelling.

\section{Spectral Clustering}
Spectral clustering is a crucial algorithm for learning on large data sources.
We will analyze Ng et al's \cite{ng2002spectral} spectral clustering paper from
NIPS 2001, which proposes a concise algorithm and provides a theoretical
backing for its choices. This paper proposes reducing the problem to regular
clustering in the eigenvector space of the Laplacian matrix of the data,
provides theoretical bounds relating the clusters in this space to clusters in
the data space, and shows initial experimental results using this algorithm.

\subsection{The Clustering Problem}
Spectral clustering aims to solve the problem of grouping large amounts of high
dimensional data into a small number of clusters. We desire a clustering
algorithm that takes as input the following:

\begin{align*}
\textbf{Ideal input: } \label{clustering:idealinput} \\
X &: \begin{pmatrix}
        x_1 \\
        x_2 \\
        \vdots \\
        x_n
     \end{pmatrix} \in \R^{n \times k} \\
s(x_i, x_j) &: \text{A real valued similarity.}
\end{align*}

and output an assignment of points to clusters as follows:

\begin{align*}
    \textbf{Ideal output: } \\
C &: \{ C_{i_1}, C_{i_2}, \cdots, C_{i_n} \} ; i \in \{1,...,k\} \\
k &: \text{Number of clusters}
\end{align*}

Unfortunately, the problem as described above is slightly ill-posed.
After all, we have not specified what kind of cluster assignments do we desire.
An algorithm which simply assigns a unique cluster index to each data point is
a valid solution, as is one that assigns every datapoint to the same cluster.
Intuitively, if we have no further information about the data points, we want
clusters to be \textit{tight} and \textit{separated}. This intuition can be more
formally described via \textit{internal} cluster evaluation metrics, such as the 
Davies-Bouldin index \cite{davies1979cluster}.

\subsubsection{Relation to graphs}
This problem, then, has a neat graph analog, as follows. Let each point $x_i$
be a node, and let edge weights be equal to the similarities between them, as
defined by $s(x_i, x_j)$. In this formulation, we desire clusters corresponding
to components that arise from \textit{sparse} cuts.

\subsection{Preliminaries}
%
% 1) Laplacian
% 2) Eigenvectors of Laplacian
% 3) Matrix Perturbation Theory

\subsection{Ng, Jordan, Weiss (NJW) Algorithm}
The Ng, Jordan, Weiss algorithm presents a spectral approach to clustering. It
takes as input $X$ as defined in the ideal input, and the following
\begin{align*}
    k &: \text{The number of clusters expected} \\
    \sigma &: \text{A scaling parameter for the similarity function.}
\end{align*}

The similarity metric is pre-defined as follows:
\begin{align*}
    s(x_i, x_j) = \exp\{\frac{\norm{x_i - x_j}^2}{2\sigma^2}\}
\end{align*}

Note that this algorithm requires advanced knowledge of the number of clusters,
an issue \achal{WC} that we will attempt to alleviate later.

The algorithm is as follows:
\begin{enumerate}
    \item Construct the matrix $A$ such that $A_{ij} = \begin{cases}s(x_i, x_j) &\;\text{ if } i \neq j \\
                                                                    0           &\;\text{ otherwise}\end{cases}$
    \item Let $D$ be $diag(A \vec{1})$. Let $\cA = D^{-1/2} A D^{-1/2}$.
    \item Find the $k$ largest eigenvectors of $\cA: v_1,...,v_k$, and form the
            matrix $V = [v_1, ..., v_k]$.
    \item Form the matrix $Y$ by normalizing the rows of $X$ to have unit length:
            $Y_{ij} = X_{ij} / \sum_j X_{ij}$.
    \item Treat $Y$ as a data matrix, so that $Y = [y_1, ..., y_n]^T$, and
        cluster these data points to obtain the cluster assignments for $y_i$:
        $C'_{i_1}, C'_{i_2}, ..., C'_{i_n}$.
    \item Let $C_{i_1} = C'_{i_1}$; that is, assign $x_i, x_j$ to the same cluster
        iff $y_i, y_j$ were assigned to the same cluster.
\end{enumerate}

\subsection{Proof sketch}


However, Ng et al. do not touch upon the computational complexity of their
method. Although a number of papers since then attempted to improve the
complexity, we focus on a recent paper by Gittens et
al.\cite{gittens2013approximate}, which proposes an approximate algorithm that
is both relatively easy to implement, and provides theoretical bounds on its
accuracy. We will analyze the algorithm that Gittens proposes, which ultimately
relies on a randomized subiteration process.


\newpage

\section{Spectral Crowdsourcing}
    Many machine learning algorithms rely on having accurate ground truth for complex data sets. These ``ground truth" labels are usually delegated to humans on a system such as Amazon Mechanical Turk. Unfortunately, without ground truth labels, it is difficult to analyze the accuracy of human labels, which in turn increases the difficulty of obtaining the labels. Due to the circular nature of this problem, there is no direct way to verify the accuracy of these labels. Much of the previous work in this field revolves around the David-Skene Estimator \cite{dawid1979maximum}. The Dawid-Skene estimator has been widely used for inferring the true labels from the noisy labels provided by non-expert crowdsourcing workers. However, since the estimator maximizes a non-convex log-likelihood function, it is hard to theoretically justify its performance. Moreover, the iterative nature of EM provides no bound on the speed of convergence.  Recent work by Zhang et al. \cite{zhang2014spectral} presents a new spectral method for ground truth label generation by using spectral methods to estimate the confusion matrix for the labelers, and provides a theoretical bound on the rate of convergence. In fact, the paper claims this method reaches very close to a global optimum in just one iterate of EM. For the sake of simplicity, we will consider only the two class classification case; the generalization to $k$ classes is natural.

\subsection{Problem Statement}
Consider an experiment where $M$ labelers assign labels to $N$ samples from the set: $\left\{[0,1],[1,0]\right\}$ (negative, positive). I will refer to these vectors as $N$ and $P$ respectively. Note that although our labels represent scalars, we represent them by the standard basis vectors of $\mathcal{R}^{2}$. This will simplify our later analysis. We will denote labeler $m$'s label of sample $n$ as $z_{mn}$.

\subsubsection{Assumptions}
\begin{enumerate}
\item
The true label $y_{i}$ of item
$i \in \left\{0,1....N\right\}$ is assumed to be sampled from a probability
distribution $p(y_{i} = 0)$ = $w_{0}$ where $w_{0}$ is the prior on label 0.

We define $W$ to be a diagonal $2 \times 2$ matrix:
\begin{align}
\begin{bmatrix}
w_{0} & 0 \\
0     & w_{1}
\end{bmatrix}
\end{align}

Naturally $w_{0} + w_{1} = 1$. \\

\item
Asssume that each of the labelers are conditionally independent given the true label. More formally:

$$p(z_{ij} = X | y_{j} = X)p(z_{hj} = X | y_{j} = X) = p(z_{ij}z_{hj} | y_{j} = X)$$

Where $i \neq h$.

Though this assumption may not hold for all instances of the labelling problem, it greatly simplifies our solution. We also assume that the vendors are more likely to get a sample correct than incorrect, and that every labeler labels every datapoint once.

\item
    Finally we assume each worker is associated with a $(2 \times 2)$ confusion matrix. Where $(l,c)$-th entry represents $P(\text{Worker Label} = l| \text{True Label} = c)$. We will denote the confusion matrix for worker $j$ as  $C_{j}$.

\end{enumerate}

Now what we want is to estimate $y_{i}$ for all $i \in \left\{0,1....N\right\}$. And estimate $C_{j}$ for all $j \in \left\{0,1....M\right\}$.
We would also like to estimate $W_{0}$ and $W_{1}$

\subsection{Dawid-Skene Estimator}
Let $\theta = (C_{0}...C_{M},w_{0},w_{1})$ be the parameters of our model.

We can define the likelihood of the data $P(z,y| \theta)$ as:

\begin{equation}  \label{eq:L}
\mathcal{L}_{d}(\theta) =
\displaystyle\prod\limits_{n=0}^{N} \displaystyle\prod\limits_{c=0}^{1}
\Big (w_{l} \displaystyle\prod\limits_{m=0}^{M} \displaystyle\prod\limits_{l=0}^{1} (C_{m}[l,c])^{z_{mn}[l]} \Big )^{y_{n}[c]}
\end{equation}

This can be easily verified using our above assumptions
We want: \\

\begin{equation} \label{eq:maxL}
\argmax_{\theta} \mathcal{L}_{d}(\theta)
\end{equation}

Unfortunately we don't know $y_{n}$ so we can't directly solve this optimization problem.
So instead we opt for an iterative algorithm with some nice theoretical guarantees.

The intuition is quite simple. There two unknown quantities of interest: $y$
and $\theta$.  Given one, the other is computable in some manner. So we will
hold one of the two unknowns constant, and compute the other. We will then hold
the other unknown constant and repeat until convergence. This algorithm is
called the Expectation Maximization (EM) algorithm.

\subsection{Expectation Maximization Algorithm}

We sum out over $y_{n}$ in \label{eq:L} by taking a
conditional expectation.

\begin{equation} \label{eq:L'}
    \mathcal{L'}(\theta) = \E_{y | z, \theta^{(t)}}[L(\theta)]
\end{equation}

Where $\theta^{(t)}$ is the estimate of $\theta$ at timestep $t$. This step
allows us to compute $y$ from a partiulcar $\theta^{(t)}$.

Next, we solve

\begin{equation} \label{eq:maxL'}
    \theta^{(t+1)} = \argmax_{\theta} \mathcal{L'}(\theta)
\end{equation}

\begin{enumerate}
    \item Pick some initial estimates for the unknown $y$ call these $y^{(0)}$
    \item Solve \eqref{eq:maxL'} with the labels $y^{(0)}$ for an optimum $\theta^{(0)}$
    \item Calculate $L'(\theta^{(0)})$ Call these labels $y^{(1)}$
    \item Repeat steps 2 and 3 until convergence
\end{enumerate}

As it turns out, this algorithm is not only simple, but also has some nice
theoretical guarantees.

\subsubsection{Analysis of EM}
Much of this analysis was borrowed from \cite{murphy2012machine}. The likelihood
of the observed data is:

\begin{equation}  \label{eq:observedL}
\mathcal{L}_{d}(\theta) =
\displaystyle\prod\limits_{n=0}^{N} \Big ( \displaystyle\sum\limits_{c=0}^{1}
w_{l} \displaystyle\prod\limits_{m=0}^{M} \displaystyle\prod\limits_{l=0}^{1} (C_{m}[l,c])^{z_{mn}[l]} \Big )^{y_{n}[c]}
\end{equation}

Unfortunately, this is very hard to optimize, which is why we took an
expectation of \eqref{eq:L}. This is the function we can optimize with the data
we are given. We will prove that the expectation we took is a lower bound for
this function.

\newpage
\begin{theorem}
\emph{Expected Likelihood is a lower bound}
\label{EMlower}
$$ \mathcal{L}_{d}(\theta) \geq E[\mathcal{L}_{d}(\theta)]$$
\begin{proof}

Let $q(y_{i})$ be the (unknown) distribution over $y_{i}$.

For simplicity's sake we will make the following definition:

\begin{equation} \label{eq:przyn}
p(z,y_{n}|\theta) :=
\Big (w_{l} \displaystyle\prod\limits_{m=0}^{M} \displaystyle\prod\limits_{l=0}^{1} (C_{m}[l,c])^{z_{mn}[l]} \Big )^{y_{n}[c]}
\end{equation}

First we will take the logarithm of \eqref{eq:observedL}

\begin{equation}
\log (\theta) = \displaystyle\sum\limits_{n=0}^{N} \log
\displaystyle\sum\limits_{c=0}^{1}
\Big (w_{l} \displaystyle\prod\limits_{m=0}^{M} \displaystyle\prod\limits_{l=0}^{1} (C_{m}[l,c])^{z_{mn}[l]} \Big )^{y_{n}[c]}
\end{equation}

Plugging in \eqref{eq:przyn}
\begin{equation} \label{eq:lll}
\log \mathcal{L}_{d}(\theta) = \displaystyle\sum\limits_{n=0}^{N} \log \displaystyle\sum\limits_{y_{n}} p(z,y_{n} | \theta)
\end{equation}

Now, multiply by $\frac{q(y_{n})}{q(y_{n})}$

\begin{equation}
\log \mathcal{L}_{d}(\theta) = \displaystyle\sum\limits_{n=0}^{N} \log \Big ( \displaystyle\sum\limits_{y_{n}} q(y_{n})\frac{p(z,y_{n} | \theta)}{q(y_{n})} \Big )
\end{equation}

The logarithm function is concave, so by Jensen's, we now obtain a lower bound
\begin{equation}\label{eq:jensens}
\log \mathcal{L}_{d}(\theta) \geq
\displaystyle\sum\limits_{n=0}^{N} \Big ( \displaystyle\sum\limits_{y_{n}}^{1} q(y_{n})\log \frac{p(z,y_{n} | \theta)}{q(y_{n})} \Big )
\end{equation}

Now we will simplify and upper bound the inner sum to show that we have a true upper bound.
\begin{equation}
\displaystyle\sum\limits_{y_{n}} q(y_{n}) \log \frac{p(z,y_{n}| \theta)}{q_{i}}
\end{equation}

\begin{equation}
\displaystyle\sum\limits_{y_{n}} q(y_{n}) \log \frac{p(y_{i}|z,\theta)p(z| \theta)}{q_{i}}
\end{equation}

\begin{equation}
-\mathcal{KL}(q(y_{n}) || p(y_{n} | z, \theta) + \log p(z| \theta)
\end{equation}

We want to saturate the lower bound, and we can do so by setting
$q(y_{n}) = p(y_{n} | x_{i}, \theta)$. This makes our KL divergence 0,
and we can now plug in this into the rhs \eqref{eq:jensens} to get

\begin{equation}
\displaystyle\sum\limits_{n=0}^{N} \log p(z | \theta)
\end{equation}

Now do a trivial transformation
\begin{equation}
\displaystyle\sum\limits_{n=0}^{N} \log \displaystyle\sum\limits_{y_{n}}  p(z,y_{n} | \theta)
\end{equation}

Which is exactly equal to \eqref{eq:lll}. Thus the highest our lower bound term can be is exactly our upper bound, thus our lower bound is truly a lower bound.

\qed
\end{proof}

\end{theorem}

This theorem is useful as it means the expectation function we are maximizing
gives us a lower bound on the true likelihood.

I will also state another theorem but not prove it. This can be verified quite
easily using the previous theorem

\begin{theorem}
\emph{EM monotonically increases the observed data log likelihood}
\end{theorem}

Now, a major issue with EM is the non convexity of the likelihood function we
are optimizing. Thus, we have little guarantees on whether we are at a local or
global maxima. Recent work by Zhang et al, promises a statistically optimal
initialization of the EM algorithm for the labeling problem previously mentioned
using some spectral methods.

\subsection{Spectral Initialization}

We will try and estimate the confusion matrix for each of the labelers. The
guarantee is our estimated confusion matrix should be ``close" to the true
confusion matrix, so that when we iterate on EM we will quickly converge to the
optimum confusion matrices for our labelers.

The high level overview of this is as follows:

\begin{enumerate}
\item Partition workers into three disjoint non empty groups $G_{1},G_{2}, G_{3}$
\item Use method of moments to estimate average confusion matrices for the three groups.
\item Use iterated expectation and independence assumption to acquire confusion matrix for each labeler
\end{enumerate}

We will denote confusion matrix column $l$ for labeler $m$ as $\mu_{ml}$

For each group we will calculate the average label for sample $n$, $g \in [3]$:

\begin{equation}
Z_{gn} = \frac{1}{|G_{g}|} \displaystyle\sum\limits\sum_{i \in G_{g}} z_{in}
\end{equation}


We will denote the average confusion matrix columns by
\begin{equation}
\mu^{\diamond}_{gl} := \E(Z_{gn} | y_{n} = l) = \frac{1}{G_{g}} \displaystyle\sum\limits_{i \in G_{g}} \mu_{il}
\end{equation}

Naturally
\begin{equation}
C^{\diamond}_{g} := [\mu^{\diamond}_{g0},\mu^{\diamond}_{g1}]
\end{equation}

We will attempt to solve for $C^{\diamond}_{g}$ from the moments of $Z_{gn}$

Before we do that, we can motivate this approach by showing its easy to acquire $C_{m}$ for all $m \in [M]$  given $C^{\diamond}_{1},C^{\diamond}_{2},C^{\diamond}_{3}$


\begin{lemma}
For $g \in [3], i \in G_{g}, a \in [3]\backslash{g}$
$C_{i} = \E[z_{ij}Z^{T}_{aj}(WC^{T{\diamond}}_{a})^{-1}$
\begin{proof}
$$ E[z_{ij}Z^{T}_{aj}] = E[E[z_{ij}Z^{T}_{aj}| y_{j}]] $$
$$ \displaystyle\sum\limits_{l = 0}^{1} w_{l}\E[z_{ij}Z^{T}_{aj} | y_{j} = l]$$

Exploit conditional independence
$$ \mathbb{E}[z_{ij}Z^{T}_{aj} | y_{j} = l] =  \E[z_{ij} | y_{j} = l] \E[Z^{T}_{aj} | y_{j} = l] = \mu_{il}(\mu^{\diamond}_{al})^{T} $$


$$ \displaystyle\sum\limits_{l = 0}^{1} w_{l}\mu_{il}(\mu^{\diamond}_{al})^{T}$$

$$\mathbb{E}[z_{ij}Z^{T}_{aj} | y_{j} = l] =  C_{i}W(C^{\diamond}_{a})^{T} $$

Take an inverse to get the desired result

$C_{i} = \E[z_{ij}Z^{T}_{aj}(WC^{T\diamond}_{a})]^{-1}$

Note: This assumes $C^{\diamond}_{a}$ is invertible, but since these are only $2 \times 2$ matrices, its not a huge constraint.
\qed
\end{proof}
\end{lemma}

Now we know that we can recover our individual confusion matrices from the averaged ones. Lets go about computing the average confusion matrices. This will hinge on some theorems that I will leave unproven, as they require much computation.

\begin{lemma}
$p(Z_{cj} = i, Z_{bj} = k) = E[Z_{cj} \otimes Z_{bj}]_{ik}$
$p(Z_{cj} = i, Z_{bj} = k, Z_{aj} = l),  = E[Z_{cj} \otimes Z_{bj} \otimes Z_{aj}]_{ikl}$

Where $E[Z_{cj} \otimes Z_{bj}]$ is a $2 \times 2$ matrix. And $E[Z_{cj} \otimes Z_{bj} \otimes Z_{aj}]$ is a $2 \time 2 \times 2$ tensor. (Here I'm using the term tensor as an n-dimensional generalization of a matrix).
\end{lemma}

Unfortunately, because the condition probabilities of our groups are not going
to be the same, neither our matrices nor our tensors will be symmetric.

Much of the averaged confusion matrix generation hinges on tensor
factorization. Tensor factorization is much more difficult than matrix
factorization, and cannot be applied to all tensors. In particular we will need
our tensor to be whitened and symmetric. So the following "diagonalizes" our
moment matrix.

\begin{theorem}
Assume that the columns of our confusion matrix are linearly indpendent. Let $(a,b,c)$ be a permutation of $[3]$.

$$Z'_{aj} := \E[Z_{cj} \otimes Z_{bj}](\E[Z_{aj} \times Z_{bj}])^{-1}Z_{aj}$$
$$Z'_{bj} := \E[Z_{cj} \otimes Z_{aj}](\E[Z_{bj} \times Z_{aj}])^{-1}Z_{bj}$$
$$M_{2} := \E[Z'_{aj} \otimes  Z'_{bj}]$$
$$M_{3} := \E[Z'_{aj} \otimes Z'_{bj} \otimes Z'_{cj}]$$

Then,
$$M_{2} = \displaystyle\sum\limits_{l=1}^{k} w_{l}\mu^{\diamond}_{cl} \otimes \mu^{\diamond}_{cl} $$
$$M_{3} = \displaystyle\sum\limits_{l=1}^{k} w_{l}\mu^{\diamond}_{cl} \otimes \mu^{\diamond}_{cl} \mu^{\diamond}_{cl} $$
\end{theorem}

This theorem is extremely powerful for two reasons:

\begin{enumerate}
\item The expectations are very easy to compute empirically from our data, so we can easily compute $M_{2}$ and $M_{3}$
\item We can use all 3 permutations of $[3]$ to generate the moment matrices and tensors for all three of our groups.
\item Our resulting structures are symmetric so now we can apply a tensor factorization algorithm to recover the columns of $C^{\diamond}_{g}$
\end{enumerate}

So now that we have the moments for our groups, we can employ a tensor factorization algorithm such as the one discussed in \cite{anandkumar2014tensor} to recover the columns of our confusion matrix, and the corresponding entries of $W$

There is one complication in that the tensor decomposition algorithm only gives us the columns that make up the original matrix but not the order of the columns. But, if we make an assumption that all the labelers have a ``good" ($>50\%$) true positive rate, we can disambiguate this by setting the index of the column to be the index of the column \textit{entry} with the highest value. If there is a collision for an index, we can break the tie by picking the one with the higher value in that slot. Again, this process is greatly simplified by the fact that we only have 2 columns in each confusion matrix.

Finally with our averaged confusion matrices we can acquire the confusion matrix for each of the labelers, and we have completed our EM initialization.

Our initialization has a very nice theoretical property in that it is close to the true confusion matrix for each labeler. Let us define some metrics so we can make this notion precise:

$$ w_{min} := min(w_{0}, w_{1}) $$
This will be the smallest portion of true labels. We assume that this value is strictly positive.

$$ \kappa := \min_{g \in [3]} \min_{l \in \left\{0,1\right\}} \min_{j \in \left\{0,1\right\}\backslash l}  \mu^{\diamond}_{gll} - \mu^{\diamond}_{glc}$$

$\kappa$ denotes the smallest gap between diagonal entries and non-diagonal entries in the confusion matrix. The assumption requires that $\kappa$ is strictly positive.

$$D_{KL} := \displaystyle\sum\limits_{i=0} p(i)\log(\frac{p(i)}{q(i)}) $$

$$\bar{D} := \min_{l \neq l'} \frac{1}{m} \displaystyle_{i=1}^{m} \mathcal{D}_{KL}(\mu_{il},\mu_{il'})$$

Let $\sigma_{L}$ be a scalar that is greater than the smallest eigenvalue of our moment matrix (between a and b).

The quantity $\bar{D}$ lower bounds the averaged KL-divergence between two columns If $\bar{D}$ is strictly positive it means that every pair of labels can be distinguished by at least one subset of workers.


\begin{theorem}
For any scalar $\delta > 0$ and any scalar $\epsilon$ satisfying $\epsilon\leq min(\frac{72\kappa}{w_{min}\sigma_{l}, 2})$ if the number of
items n satisfies $ n = \Omega\Big ( \frac{(32\log((m + 2){\delta})}{\epsilon^{2}\omega^{2}\sigma_{L}^{13}} \Big ) $

Then the confusion matrices returned by our initialization is bounded as

$$ ||\hat{C_{i}} - C_{i}||_{\infty}  < \epsilon \text{for all} \; i \in  [m] $$

The proof for this theorem is rather long, and requires extensive computations,
bounding and algebra. It is omitted for brevity and clarity, but can be found
in \cite{zhang2014spectral}.
\end{theorem}

The neat fact about this theorem is that the non-convexity of the EM algorithm
does not impact us, as for large $n$ we start very close to the actual
maxima.


\bibliographystyle{unsrt}
\bibliography{ref}

\end{document}
